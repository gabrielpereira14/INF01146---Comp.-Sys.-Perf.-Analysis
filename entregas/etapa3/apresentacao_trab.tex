% Created 2025-11-27 Thu 23:07
% Intended LaTeX compiler: pdflatex
\documentclass[presentation]{beamer}
\usepackage[utf8]{inputenc}
\usepackage[T1]{fontenc}
\usepackage{graphicx}
\usepackage{longtable}
\usepackage{wrapfig}
\usepackage{rotating}
\usepackage[normalem]{ulem}
\usepackage{amsmath}
\usepackage{amssymb}
\usepackage{capt-of}
\usepackage{hyperref}
\setbeamertemplate{itemize item}{\textbullet}
\setbeamertemplate{itemize subitem}{\textbullet}
\setbeamertemplate{itemize subsubitem}{\textbullet}
\setbeamertemplate{headline}{}
\setbeamertemplate{footline}{}
\setbeamertemplate{navigation symbols}{}
\usetheme{default}
\author{Ester Crestani, Gabriel Pereira e Júlia Pimentel}
\date{30/11/2025}
\title{Medição do impacto de uma infraestrutura de VPN na taxa de transmissão e na latência}
\hypersetup{
 pdfauthor={Ester Crestani, Gabriel Pereira e Júlia Pimentel},
 pdftitle={Medição do impacto de uma infraestrutura de VPN na taxa de transmissão e na latência},
 pdfkeywords={},
 pdfsubject={},
 pdfcreator={Emacs 29.3 (Org mode 9.6.10)}, 
 pdflang={English}}
\begin{document}

\maketitle

\begin{frame}[label={sec:orgec554cb}]{Objetivos do trabalho}
\begin{itemize}
\item Avaliar o impacto da infraestrutura de VPN da UFRGS
\end{itemize}
\end{frame}

\begin{frame}[label={sec:orgae1913a}]{Metodologia}
Execução simultânea de ping e iperf através de um script que alterna entre:
\begin{itemize}
\item Teste com a conexão via OpenVPN UFRGS.
\item Teste com conexão direta.
\end{itemize}
\end{frame}

\begin{frame}[label={sec:orga638465},fragile]{Dados coletados}
 \begin{itemize}
\item Coleta automática de: latência, jitter, perda de pacotes (ping), throughput TCP e UDP (iperf3)
\item Dados coletados por integrante do grupo, disponiveis dentro da pasta data
\item Arquivos: 
\begin{itemize}
\item \texttt{ping\_results.csv}
\item \texttt{iperf\_results.csv}
\item \texttt{iperf\_results\_reversed.csv}
\end{itemize}
\end{itemize}
\end{frame}

\begin{frame}[label={sec:org189569c}]{Ambiente de testes}
\begin{itemize}
\item Servidor iperf: pcad.inf.ufrgs.br
\item Site para ping: moodle.ufrgs.br
\end{itemize}

Autenticação automática via arquivo pass.txt
Intervalos entre testes: 5 segundos
Duração de cada medição: 10 segundos

\begin{itemize}
\item Testes realizados em diferentes dias e horários para melhor amostragem
\end{itemize}
\end{frame}

\begin{frame}[label={sec:orgf873516}]{Material para medição}
\end{frame}

\begin{frame}[label={sec:org6c4256a}]{Resultados preliminares:}
\begin{itemize}
\item Gráficos
\end{itemize}
\end{frame}

\begin{frame}[label={sec:orgc0559a0}]{Próximos passos:}
\begin{itemize}
\item Realizar mais medições em diferentes horários
\item Gerar novas análises e indicadoresP
\item Preparar análise crítica para etapa final
\end{itemize}
\end{frame}

\begin{frame}[label={sec:org824851c}]{Slide 3 exemplo para imagem}
\begin{center}
\includegraphics[width=1\textwidth]{/home/gabip/perf/gabriel/images/latency_gabriel.png}
\end{center}
\end{frame}
\end{document}